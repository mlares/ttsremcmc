%                                                                 aa.dem
% AA vers. 8.1, LaTeX class for Astronomy & Astrophysics
% demonstration file
%                                                       (c) EDP Sciences
%-----------------------------------------------------------------------
%
%\documentclass[referee]{aa} % for a referee version
%\documentclass[onecolumn]{aa} % for a paper on 1 column  
%\documentclass[longauth]{aa} % for the long lists of affiliations 
%\documentclass[rnote]{aa} % for the research notes
%\documentclass[letter]{aa} % for the letters 
%
\documentclass{aa}  
                       

%
\usepackage{graphicx}
%%%%%%%%%%%%%%%%%%%%%%%%%%%%%%%%%%%%%%%%
\usepackage{txfonts}
%%%%%%%%%%%%%%%%%%%%%%%%%%%%%%%%%%%%%%%%
%\usepackage[options]{hyperref}
% To add links in your PDF file, use the package "hyperref"
% with options according to your LaTeX or PDFLaTeX drivers.
%
\begin{document} 

%%%%%%%%%%%%%%%%%%%%%%%%%%%%%%%%%%%%%%%%%%%%%%%%%%%%%%%%%%%%·B·E·G·I·N·

   \title{A Bayesian approach to the study of stellar formation}

   \subtitle{}

   \author{L. Gramajo
          \inst{1}
          \thanks{luciana@oac.uncor.edu}
          \and
          M. Lares
          \inst{1,2}\fnmsep
          }

   \institute{Observatorio Astron\'omico, Universidad Nacional de C\'ordoba \\
              \email{luciana@oac.uncor.edu}
         \and
              Instituto de Astronom\'{\i}a Te\'orica y Experimental (IATE)
             }

             
   \date{Received September 15, 1996; accepted March 16, 1997}

% \abstract{}{}{}{}{} 
% 5 {} token are mandatory
 
  \abstract
  % context heading (optional)
  % {} leave it empty if necessary  
   {Context}
  % aims heading (mandatory)
   {Aims}
  % methods heading (mandatory)
   {Methods}
  % results heading (mandatory)
   {Results}
  % conclusions heading (optional), leave it empty if necessary 
   {}

   \keywords{stellar formation -- statistical methods }

   \maketitle



%····································································
%                                00SECTIONS

\section{Introduction} \label{S_intro}
%{{{


%% algo de estrellas, formacion estelar, etc.


The study of many astronomical phenomena is characterized by the inability
to control experiments.
%
Given an object under study, its data are unique and unrepeatable, and
we rely on the development of models to assess the nature of the
underlying physical processes.
%
On the development of models, a series of competing models are
compared to the available data and the best model must be addressed by
means of statistical inference.
%
Bayesian inference is one of the most used techniques in astronomy and
physics to obtain the most reliable model to represent observational
data.



%}}}

\begin{figure}
   \centering
   \includegraphics[width=0.5\textwidth]{figs/plot.pdf}
   \caption{
   Distribution of concentration, $D_{4000}$, SFR and starmass for galaxies.
   }
   \label{F_D4000}
\end{figure}





\section{Case study: ISO Oph 32} \label{S_stats}
%{{{
 
%}}}

\section{MonteCarlo Methods} \label{S_MCMC}
%{{{
 
In this section we describe the methodology used to derive constraints
on the stellar formation model.
%
We use the model developed by Whitney et al (200?), hereafter W07.
%
This model has several free parameters that can be adjusted to fit
observed data.
%
We implement a Markov Chain Monte Carlo (MCMC) procedure to traverse
the parameter space. 
%
Although the model has 15 parameters, some of them are well known for
this particular object, so that we fixed them and explored a subspace
of the parameters.
%
The subset of parameters chosen to be explored comprise the following
parameters: 



The parameter space explored is limited to 16 parameters, as described
below:
 
\begin{itemize}
   \item Central source parameters: stellar mass, radius, temperature
   \item Infalling envelope parameters: envelope mass acretion rate,
      envelope outer radius, cavity opening angle
   \item Disk parameters: disk mass, disk outer radius, disk inner
      radius, disk mas accretion rate, disk radial density exponent
\end{itemize}
 




The goal is to find the model that maximizes the posterior probability
for a given data.
%
This quantity is related to the likelihood, i.e., the probability of
the data given the model, andthe priors, which is the joint
probability of the model, and is given by the Bayes formula:


\begin{equation}
P(M|D) = \frac{P(D|M)}{P(M)}
\end{equation}

The prior functions are given ad-hoc for each case study, and comprise
information from observations, bibliography or previous measurements.
%
For example, when studiyng a proto-stellar object confirmed to be in
the FU Orionis family, the general properties or range of values for
the parameters are known.
%
The incorporation of the prior function does not affect the final
probability distribution, but enhances the ability of the system to
find the optimal region in parameter space in less time.



This method is widely used to contrain model parameters (citar
Spergel, Sanchez, Stroeer).
%
One of their most prominent properties is that has the advantage of
allowing for the computation of a confidence limits for each
parameter.
%
Also, it allows for an automated, unbiased way to find the best model
for a given set of observations of a proto-stellar object.
%
The information gathered from observations or in the bibliography are
incorporated to the model finding pipeline in the definition of the
priors for the model parameters.
%
%}}}




\section{Statistical Analysis} \label{S_stats}
%{{{     


\subsection{The choice of priors} \label{S_}

Criterios para elegir los priors.  Tipo de objeto, mediciones
anteriores, bibliografia.

\subsection{Exploring the parameter space } \label{S_}

Eleccion del sigma de los saltos...

\subsection{Convergence of Markov Chains} \label{S_}

criterio de convergencia de metropolis hastings

\subsection{Confidence intervals for the best model} \label{S_}

calculo de los errores

%}}}




\section{Conclusions} \label{S_conclusus} 
%{{{
 
%}}}


%····································································


\footnotesize{
  \bibliographystyle{mn2e}
  \bibliography{Bibliography}
}

\end{document}



